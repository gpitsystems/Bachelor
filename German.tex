\subsection{Abwägung zwischen Open-Source und Cloud-Vendor}
% Wie bereits im vorherigen Abschnitt beschrieben, 
% OpenFaaS  
Die Abwägung zwischen der Implementation von Function as a Service in der unternehmenseigenen Cloud bzw. Servern oder dem Outsourcen an einen proprietären Cloud-Vendor, sollte wohl überlegt sein, da hiervon in der Folge eine Vielzahl an Möglichkeiten und Restriktionen abhängt. \\\\
Entscheidet man sich für ersteres, also dem Aufbau einer privaten FaaS-Plattform so stehen hierfür, mit IBM Apache OpenWhisk, Fission, OpenFaaS oder Kubless, nahezu identisch viele Frameworks zur Verfügung wie bei den öffentlichen Vendoren. \textcolor{blue}{Schaubild CNCF Serverless Landscape ggf. einfügen.} Des Weiteren bietet dieser Weg, abgeseh en von der Vielfalt an Frameworks mit wiederum unterschiedlichen Eigenschaften, die Möglichkeit die Größe der Funktionen oder die maximal zulässige Laufzeit eines Containers individuell anzupassen. Dies würde einerseits eine granularer Verrechnung der in Anspruch genommenen Ressourcen der einzelnen Bereiche oder Teams zulassen, zugleich aber auch ein Operations-Team verlangen, welches sich in das jeweilige Framework einarbeitet, die Plattform zu dessen Betrieb aufsetzt und die spätere Wartung dieser übernimmt. \cite{mohanty2018evaluation}.\\\\
Eine Alternative stellen proprietäre Lösungen von öffentlichen Cloud-Vendoren, wie beispielsweise Amazon Web Services (AWS) Lambda, Microsofts Azure Functions, IBM Cloud Functions oder Google Cloud Functions, bei welchen sich von Unternehmensseite aus niemand um die Wartung der Infrastruktur, die Skalierung der Services, die Behandlung von Fehlermeldungen o.ä. kümmern muss. Diese Aufgaben werden in der Folge von dem Plattformbetreiber übernommen. Natürlich sind diese Dinge in erster Linie Aufgabe des Plattformbetreibers, jedoch haben sie unmittelbare Auswirkungen auf das Unternehmen, sollten Probleme beim Skalieren von Funktionen oder dem Monitoring auftreten. Sollte dies der Fall sein, so sind die Entwickler von den bereitgestellten Debugging Möglichkeiten und Monitoring-Lösungen der Plattform abhängig um Probleme schnellstmöglich zu beheben, sollte dies die Plattform nicht tun. Daneben ist ein weiterer Punkt des Vendor Lock-Ins die von Anbieter zu Anbieter variierende Infrastruktur, welche ein einfaches Shiften von Funktionen erheblich erschwert. Zudem sind die Benutzung der Funktionen meist automatisch mit der Inanspruchnahme weiterer Services der Plattform, wie dem Message Queuing oder der Datenspeicherung, gekoppelt. \\\\
Dies hat sowohl Vor- als auch Nachteile. Ist man bei der Einrichtung der privaten FaaS-Cloud auf die unternehmensinternen Ressourcen beschränkt, so bieten die Cloud-Anbieter, neben den Restriktionen des Vendor Lock-Ins, in den meisten Fällen ein großes Ökosystem an weiteren Services, welche sich problemlos an die Funktionen anbinden lassen. Im Folgenden wird daher, auch wenn sich diese Arbeit hauptsächlich auf FaaS beschränkt, das Ökosystem der einzelnen Cloud-Anbieter kurz betrachtet, um einen besseren Überblick über deren zusätzliche Leistungen zu erhalten und eine fundierte Entscheidung treffen zu können.


