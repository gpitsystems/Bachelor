\subsection{Abwägung zwischen Open-Source und Cloud-Vendor}
% Wie bereits im vorherigen Abschnitt beschrieben, 
% OpenFaaS  
Die Abwägung zwischen der Implementation von Function as a Service in der unternehmenseigenen Cloud bzw. Servern oder dem Outsourcen an einen proprietären Cloud-Vendor, sollte wohl überlegt sein, da hiervon in der Folge eine Vielzahl an Möglichkeiten und Restriktionen abhängt. \\\\
Entscheidet man sich für ersteres, also dem Aufbau einer privaten FaaS-Plattform so stehen hierfür, mit IBM Apache OpenWhisk, Fission, OpenFaaS oder Kubless, nahezu identisch viele Frameworks zur Verfügung wie bei den öffentlichen Vendoren. \textcolor{blue}{Schaubild CNCF Serverless Landscape ggf. einfügen.} Des Weiteren bietet dieser Weg, abgeseh en von der Vielfalt an Frameworks mit wiederum unterschiedlichen Eigenschaften, die Möglichkeit die Größe der Funktionen oder die maximal zulässige Laufzeit eines Containers individuell anzupassen. Dies würde einerseits eine granularer Verrechnung der in Anspruch genommenen Ressourcen der einzelnen Bereiche oder Teams zulassen, zugleich aber auch ein Operations-Team verlangen, welches sich in das jeweilige Framework einarbeitet, die Plattform zu dessen Betrieb aufsetzt und die spätere Wartung dieser übernimmt. \cite{mohanty2018evaluation}.\\\\
Eine Alternative stellen proprietäre Lösungen von öffentlichen Cloud-Vendoren, wie beispielsweise Amazon Web Services (AWS) Lambda, Microsofts Azure Functions, IBM Cloud Functions oder Google Cloud Functions, bei welchen sich von Unternehmensseite aus niemand um die Wartung der Infrastruktur, die Skalierung der Services, die Behandlung von Fehlermeldungen o.ä. kümmern muss. Diese Aufgaben werden in der Folge von dem Plattformbetreiber übernommen. Natürlich sind diese Dinge in erster Linie Aufgabe des Plattformbetreibers, jedoch haben sie unmittelbare Auswirkungen auf das Unternehmen, sollten Probleme beim Skalieren von Funktionen oder dem Monitoring auftreten. Sollte dies der Fall sein, so sind die Entwickler von den bereitgestellten Debugging Möglichkeiten und Monitoring-Lösungen der Plattform abhängig um Probleme schnellstmöglich zu beheben, sollte dies die Plattform nicht tun. Daneben ist ein weiterer Punkt des Vendor Lock-Ins die von Anbieter zu Anbieter variierende Infrastruktur, welche ein einfaches Shiften von Funktionen erheblich erschwert. Zudem sind die Benutzung der Funktionen meist automatisch mit der Inanspruchnahme weiterer Services der Plattform, wie dem Message Queuing oder der Datenspeicherung, gekoppelt. \\\\
Dies hat sowohl Vor- als auch Nachteile. Ist man bei der Einrichtung der privaten FaaS-Cloud auf die unternehmensinternen Ressourcen beschränkt, so bieten die Cloud-Anbieter, neben den Restriktionen des Vendor Lock-Ins, in den meisten Fällen ein großes Ökosystem an weiteren Services, welche sich problemlos an die Funktionen anbinden lassen. Im Folgenden wird daher, auch wenn sich diese Arbeit hauptsächlich auf FaaS beschränkt, das Ökosystem der einzelnen Cloud-Anbieter kurz betrachtet, um einen besseren Überblick über deren zusätzliche Leistungen zu erhalten und eine fundierte Entscheidung treffen zu können.


% Fig. 17. Testing approaches for FaaS functions. tions, but are also determined by what languages are made avail able. For example, of the 8 responses marked ”Other” in Fig. 16 , 5 include ”Go”, which became available in Google’s FaaS offering only when our survey was already live. Development challenges. Given the relative immaturity of the tech nology, it is unsurprising that we have observed some challenges and grievances that even advanced practitioners currently struggle with. A major challenge is how to test functions. Due to the rela tively small size and often low complexity of individual functions, they lend themselves well for unit tests, which can be performed locally. However, testing the integration of multiple functions or external services is harder, as local replication of the entire system is often not possible or hard to achieve. “[... ] it is not possible to replicate a serverless or cloud system on your local machine.” I6. One possible solution is to test functions directly in production, or in a dedicated development environment that is also hosted in the cloud. One common way to implement the latter is to have multiple separate accounts with the cloud provider, one for pro duction and one for development and testing. Both approaches have the obvious disadvantage that they require developers to pay for test invocations the same as for production workload. In ad dition, we have observed that testing in actual production envi ronments can have (negative) side-effects on production systems in some cases. One approach to deal with this issue is to perform canary releases or A/B testing, so that possible side-effects can be assessed for a small number of requests. The testing practices used by survey respondents are illustrated in Fig. 17 . As expected, unit tests are commonly performed locally. When it comes to integration tests, dedicated development envi ronments and mocked environments are more commonly used for testing than production environments (in general, or via canary re leases or A/B tests). 23.7\% (22) of respondents to this question per form tests in both, dedicated FaaS development environments and mocked FaaS environments, while only 16.1\% (15) respondents test both in a dedicated environment (dev or mocked) and in a produc tion environment. nother ore challenge is a lack of tooling and insufficient documentation. Tooling is especially desirable for the interviewees when it comes to deploying (sets of) functions, mapping events to functions (using, for example, API gateways to make functions accessible to HTTP requests), and monitoring and logging. At the same time, only a few of the available tools are actually used. With 79.7\% of survey respondents using it, the Serverless framework is by far the most common among them. Contrary, the next frequently named library, Chalice, was only named by 11.6\% of respondents. This indicates that existing tooling, with the exception of the Serverless framework, appear to not address the core challenges that developers currently face, or their existance is not yet widely known. Fig \cite{leitner2019mixed}



%The orchestration component defines the sequence of tasks; all executions are automatically triggered, each step is tracked and retried in the case of error \cite{werner2018serverless}. AWS Step Functions Each step of an application is triggered, tracked and even retried when errors occur, assuring that applications execute as expected and in the correct order. To aid debugging and diagnostics, logs are kept for every step of the process. Figure 3 shows this process. An added benefit is that step functions can be created and deployed in code through the definition of a state machine using Amazon’s JSON-based States Language10. \cite{werner2018serverless}.


% Kein garantie, dass der folgende Funktionsaufruf auf die selbe bereits laufende Instanz einer Funktion trifft, welche dann wiederum auch zugriff auf dem im Memory gespeicherten State hat. Es ist daher notwendig, dass der Application-State extern gespeichert wird. Auf diesen sog. "\textit{Shared memory}" müssen in der Folge alle Funktionen drauf zugreifen, wenn Applikations-Daten benötigt werden.


% no access to the underlying OS and install agent or daemon to gather metrics isn’t possible. 
% Moreover, the serverless function cannot be used to do some actions after the request is processed, all actions should be finished before returning response because the container is suspended as soon as a response is Moreover, the serverless function cannot be used to do some actions after the request is processed, all actions should be   before returning response because the container is suspended as soon as a response is returned. Therefore, monitoring increases request time. Asynchronous monitoring is possible by-passing metrics as log message and afterward process logs and sends metrics. Such an approach does not prolong requests however it adds delay to monitoring, increases costs due to additional logs processing and increases concurrently executed functions. Maximum limit of concurrently executed functions can be reached faster due processing of logs. Therefore, it’s better to use third-party service, such as Datadog15, when complex log analysis is needed. 
% When incorporating an open surce framework, anotehr layer will be added to the environment to display the different solutions. metrics, change provider when one has too many probelems.